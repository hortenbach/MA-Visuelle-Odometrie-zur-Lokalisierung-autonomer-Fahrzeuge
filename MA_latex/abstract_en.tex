%%
%% Abstract
%%
%%%%%%%%%%%%%%%%%%%%%%%%%%%%%%%%%%%%%%%%%%%%%%%%%%%%%%%%%%%%%%%%%%%%%


\section*{Abstract}
In the present thesis "Visual odometry for the localization of autonomous vehicles", the driven trajectory of a vehicle on test routes was estimated exclusively using camera data. The test drives were carried out under different light and weather conditions with distances of up to 2 km.
\newline

A stereo camera system and an RGB-D camera system were mounted on the vehicle and recorded together. Vehicle pose was estimated for both camera systems using the same feature-based visual odometry. Projective geometry and the extraction of image features are therefore among the most important topics of the work.
\newline 

The CARLA simulator was used as a test environment. Estimated odometry data is published on a ROS2 network, making the implementation highly modular and easily integrated into ROS2 based systems.
\newline 

Overall, orientation and position could be reliably estimated using visual odometry, with the system using the data from the RGB-D camera system achieving lower estimation errors.


%% eof
